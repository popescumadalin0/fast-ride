\documentclass[a4paper,12pt]{report}
\usepackage[a4paper, top=2.54cm, bottom=2.54cm, left=2.54cm, right=2.54cm]{geometry}
\usepackage{newtxtext,newtxmath}
\usepackage{graphicx}
\usepackage{fancyhdr}
\usepackage{setspace}
\usepackage{array}
\usepackage{ragged2e}
\usepackage{xcolor}
\usepackage[utf8]{inputenc}
\usepackage[T1]{fontenc}
\usepackage[romanian]{babel}

\justifying
\onehalfspacing % 1.5 linii
\setlength{\parindent}{1.27cm}

\newcommand{\headerpage}[1]
{
\begin{center}
  \begin{tabular}{m{2cm} m{10cm} m{2cm}}
    \includegraphics[width=1.7cm]{left-logo.png} &
    \centering
    {\fontsize{10pt}{10pt}\selectfont
      \textbf{UNIVERSITATEA DIN CRAIOVA} \\[0.3em]
      \textbf{FACULTATEA DE AUTOMATICĂ, CALCULATOARE ȘI ELECTRONICĂ} \\[0.7em]
      \textbf{DEPARTAMENTUL DE CALCULATOARE ȘI TEHNOLOGIA INFORMAȚIEI}
    } &
    \includegraphics[width=1.7cm]{right-logo.png}
  \end{tabular}
\end{center}

\vspace{4cm} % spațiu până la mijlocul paginii

% Titlul lucrării
\begin{center}
  {\fontsize{14pt}{14pt}\selectfont
   \textbf{#1} \\[1em]
  {Popescu Constantin-Mădălin}
  }
\end{center}

\vspace{3cm}

% Text mai jos
\begin{center}
   {\fontsize{12pt}{12pt}\selectfont
   \textbf{COORDONATOR ȘTIINȚIFIC} \\[1em]
   {Asist. Dr. Cătălin Cerbulescu}
   }
\end{center}

% Spațiu până jos
\vfill

% Data
\begin{center}
  {\fontsize{12pt}{12pt}\selectfont Mai 2025 \\
  Craiova}
\end{center}
}

\begin{document}
\thispagestyle{empty}
\headerpage{LUCRARE DE LICENȚĂ}

\newpage
\thispagestyle{empty}
\headerpage{FastRide. Aplicație de ride sharing}

\newpage
\thispagestyle{empty}
\null
\newpage

\pagenumbering{arabic}
\setcounter{page}{4}
\pagestyle{fancy}

\fancyhf{}
\fancyhead[L]{Popescu Constantin-Mădălin}
\fancyhead[R]{\nouppercase{\rightmark}}
\renewcommand{\headrulewidth}{0pt}
\fancyfoot[C]{\thepage}

\tableofcontents
\clearpage

\chapter{INTRODUCERE}
\section{Scopul principal}
\section{Motivație}

\newpage
\chapter{TEHNOLOGII ȘI FRAMEWORK-URI FOLOSITE}
\section{C Charp}
\section{.NET}
\section{Azure Function}
\section{Azure Table Storage}
\subsection{Azurite}
\section{LINQ}
\section{HyperText Markup Language (HTML)}
\section{Cascading Style SHeets (CSS)}
\section{Bootstrap}
\section{Blazor}
\section{MudBlazor}
\section{SignalR}
\subsection{WebSockets}
\subsection{SignalR Server Emulator}
\section{Refit API}
\section{Leaflet și Routing Machine}
\section{Stripe}
\newpage
\chapter {ELEMENTE SOFTWARE FOLOSITE}
\section{JetBrains Rider IDE}
\section{Microsoft Azure Storage Explorer}
\section{Docker}
\newpage
\chapter{SPECIFICAȚII ȘI REPREZENTAREA APLICAȚIEI}
poze doar cu cod
\section{Arhitectura aplicației}
\subsection{Stocare}
ER si relation diagram
ce tabele avem si cum populam
\subsection{Server-side}
durable orchestrations si cum functioneaza activitatile si restul
\subsection{Client-side}
blazor wasm si cum merge, cum apeleaza backendul
despre map si background services
\section{Comunicarea între componente}
diagrama de cum comunica aplicatia
\subsection{Autentificare și autorizare}
\subsection{Pagina admin-ului}
\chapter{UTILIZAREA APLICAȚIEI}
poze doar cu UI
\section{Instalarea aplicației}
ce este in readme
\section{Manual de utilizare}
User-flows
\subsection{Experiența utilizatorului}
user case flow
\subsection{Experiența șoferului}
driver case flow
\subsection{Experiența admin-ului}
admin case flow
\newpage
\chapter{CONCLUZII}
\newpage
\chapter{BIBLIOGRAFIE}


\end{document}