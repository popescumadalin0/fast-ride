\chapter {ELEMENTE SOFTWARE FOLOSITE}
\section{JetBrains Rider IDE}

JetBrains Rider este un mediu de dezvoltare integrat (IDE) dezvoltat de compania JetBrains, binecunoscută pentru IntelliJ IDEA și alte instrumente populare dedicate dezvoltatorilor. Rider a fost lansat oficial în 2017, având ca obiectiv principal să ofere o alternativă performantă și multiplatformă pentru dezvoltarea aplicațiilor .NET, compatibilă cu Windows, macOS și Linux. \parencite{rider}

Rider combină motorul de analiză a codului de la ReSharper (foarte cunoscut în rândul utilizatorilor de Visual Studio) cu platforma IntelliJ, rezultând astfel un IDE rapid, stabil și bogat în funcționalități. Este folosit pe scară largă în dezvoltarea de aplicații .NET Core, ASP.NET, Xamarin, Unity și Blazor. \parencite{rider}

Una dintre trăsăturile importante ale JetBrains Rider este faptul că este un IDE orientat pe comenzi rapide de la tastatură (keyboard-centric). Aproape orice acțiune poate fi executată rapid fără a naviga prin meniuri, iar comenzile implicite pot fi personalizate. Rider oferă o varietate de “keymaps” predefinite (configurații de comenzi rapide), pentru a se potrivi stilului fiecărui dezvoltator. \parencite{rider}

Interfața principală a aplicației este simplă și curată, dar oferă acces rapid la cele mai importante funcționalități: construirea soluției, configurarea modului de rulare/debug, integrarea cu sisteme de control al versiunilor (Git, SVN), precum și funcționalități de căutare globală sau localizare a fișierelor. \parencite{rider}

Rider este capabil să lucreze nativ cu tehnologii precum Docker, baze de date SQL, debug remote, testare unitară și CI/CD. Este compatibil cu majoritatea toolurilor moderne folosite în dezvoltare și o viteza de analizează a proiectelor mari ridicată. \parencite{rider}

În proiectul Fast Ride, JetBrains Rider a fost utilizat ca IDE principal pentru dezvoltarea backend-ului scris în C\# și gestionarea componentelor Azure Functions.

\section{Microsoft Azure Storage Explorer}
Azure Storage Explorer este un instrument grafic, dezvoltat de Microsoft, care permite
accesul și gestionarea datelor stocate în Azure Storage, fie că este vorba despre conturi de stocare,
containere, fișiere, tabele sau cozi de mesaje. Lansat pentru a simplifica interacțiunea cu serviciile
de stocare din Azure, Azure Storage Explorer oferă o interfață ușor de utilizat pentru dezvoltatori
și administratori, permițându-le să gestioneze eficient datele și resursele de stocare în cloud. \parencite{azureStorageExplorer}

Azure Storage Explorer a fost lansat de Microsoft în 2015, ca un instrument destinat
dezvoltatorilor și profesioniștilor IT care utilizează serviciile de stocare oferite de Azure. Înainte
de apariția acestuia, interacțiunea cu Azure Storage se realiza în principal prin intermediul Azure
Portal, folosind interfețe web sau scripturi și API-uri. Pentru a oferi o soluție mai intuitivă și
accesibilă pentru gestionarea resurselor de stocare, Microsoft a dezvoltat Azure Storage Explorer,
un client desktop disponibil pentru Windows, macOS și Linux. \parencite{azureStorageExplorer}

Acesta a evoluat de-a lungul timpului, integrându-se cu alte servicii din Azure și oferind
funcționalități extinse, cum ar fi gestionarea datelor offline, suport pentru acces la mai multe
conturi de stocare și securizarea accesului prin autentificare bazată pe Azure Active Directory
\(AAD\). \parencite{azureStorageExplorer}

Pentru proiectul prezent, această aplicație facilitează navigarea developer-ului prin baza de date a lucrării.

\section{Docker}

Docker Desktop este o aplicație ușor de instalat, disponibilă pentru Windows, macOS și Linux, care permite dezvoltatorilor să creeze, să partajeze și să ruleze aplicații containerizate sau microservicii cu doar câteva clicuri. \parencite{docker}

Unul dintre principalele avantaje ale Docker Desktop este interfața grafică simplă și intuitivă (GUI), care face ca gestionarea containerelor, a aplicațiilor și a imaginilor să fie accesibilă chiar și pentru cei fără experiență avansată în administrarea de medii virtualizate. Astfel, dezvoltatorii pot urmări ce rulează local, pot porni/opri containere sau pot inspecta fișierele și logurile cu ușurință, direct din interfață. \parencite{docker}

Prin utilizarea Docker Desktop, se elimină o mare parte din timpul pierdut pe configurări complexe. Aplicația gestionează automat detalii precum porturile, sistemul de fișiere, rețelele interne și alte setări implicite, oferind un mediu de dezvoltare stabil și actualizat constant cu patch-uri de securitate și corecturi de erori. \parencite{docker}

Un alt avantaj major este integrarea nativă cu majoritatea limbajelor și uneltelor de dezvoltare moderne. În plus, Docker oferă acces la Docker Hub, o platformă online cu imagini oficiale și comunitare, pe care dezvoltatorii o pot folosi pentru a construi rapid prototipuri, a automatiza procesul de build sau a implementa fluxuri de CI/CD (Continuous Integration / Continuous Deployment). \parencite{docker}

În proiectul Fast Ride, Docker a fost folosit pentru a hosta storage-ul aplicației (Azurite).
