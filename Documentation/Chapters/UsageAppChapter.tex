\chapter{UTILIZAREA APLICAȚIEI}

\section{Instalarea aplicației}

Pentru a putea folosi aplicația trebuie să fie instalate mai multe componente:
\begin{itemize}
    \item \textit{Docker} - pentru storage
    \item \textit{dotnet tool} - pentru comenzi în terminal și compilarea .NET;
    \item \textit{Azurite} - emulatorul pentru Azure Storage;
    \item \textit{SignalR Emulator} - emulatorul pentru server-ul SignalR;
    \item \textit{Azure Functions Tools} - pentru pornirea proiectului Azure;
    \item \textit{ngrok/Visual Studio sau Rider} - în caz de deployment sau doar un IDE pentru rularea codului.
\end{itemize}

Prima dată se obține codul, fie din surse, fie de pe GitHub.

Trebuie instalat \textit{Docker} deoarece \textit{Azurite} vine cu foarte multe fișiere temporale și
există riscul ca mașina să rămână fără memorie.

Următorul pas reprezintă deschiderea unui \textit{Terminal}, și să se navigheze către folderul \textit{Docker} din proiect, folosind
comanda:
\begin{verbatim}
    cd C:\Work\FastRide\fast-ride\Run\Docker
\end{verbatim}
Apoi trebuie rulată următoarea comadă pentru a instala \textit{Azurite}:
\begin{verbatim}
    docker-compose up -d
\end{verbatim}
Dacă există eroare privind comanda \textit{docker-compose}, atunci acesta trebuie instalat.
Urmează să se instaleze \textit{dotnet tool} și să se instaleze \textit{SignalR Emulator}-ul prin comanda:
\begin{verbatim}
    dotnet tool install -g Microsoft.Azure.SignalR.Emulator
\end{verbatim}

Mai departe trebuie deschis \textit{Docker}-ul și pornit \textit{Azurite}, apoi rularea comenzii:

\begin{verbatim}
    cd C:\Work\FastRide\fast-ride\Run\Docker
    asrs-emulator start
\end{verbatim}

Pentru a rula codul, fie se deschide un IDE cu ambele soluții și se rulează, fie se rulează din terminal
cu beneficiul de a publica în rețea aplicația. Pentru terminal trebuie rulate următoarele comenzi:

\begin{verbatim}
    cd C:\Work\FastRide\fast-ride\FastRide.Server
    func start

    ngrok http http://localhost:7102 # pentru ngrok (din ngrok.exe)
\end{verbatim}

Dacă se folosește \textit{ngrok} atunci trebui actualizată configurația din Client pentru \textit{SDK}-ul server-ului.
\begin{verbatim}
    cd C:\Work\FastRide\fast-ride\FastRide.Client
    dotnet run --urls "http://localhost:7028"
    ngrok http https://localhost:7028 # pentru ngrok (din ngrok.exe)
\end{verbatim}


\section{Manual de utilizare}
User-flows
\subsection{Experiența utilizatorului}
user case flow
\subsection{Experiența șoferului}
driver case flow
\subsection{Experiența admin-ului}
admin case flow