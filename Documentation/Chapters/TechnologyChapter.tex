\chapter{TEHNOLOGII ȘI FRAMEWORK-URI FOLOSITE}
\section{C\#}

C\# este un limbaj de programare modern, orientat pe obiect, dezvoltat de Microsoft și lansat în anul 2000, ca parte a inițiativei .NET. A fost creat sub conducerea lui Anders Hejlsberg, același care a contribuit la dezvoltarea limbajului Delphi. Scopul principal al C\# a fost să combine puterea și flexibilitatea limbajelor precum C++ cu simplitatea și siguranța limbajelor moderne, cum ar fi Java. \parencite{charp}

C\# este un limbaj conceput pentru a fi ușor de învățat și folosit, dar suficient de robust pentru aplicații complexe. Este ideal pentru dezvoltarea de aplicații desktop, web, mobile, jocuri (cu Unity), dar și servicii cloud, microservicii și funcții serverless, în special atunci când este combinat cu platforma .NET. \parencite{charp}

Limbajul C\# este orientat pe obiect, adică permite organizarea codului în clase și obiecte, facilitând reutilizarea și întreținerea codului. De asemenea, are suport nativ pentru asincronism (prin async/await), LINQ pentru interogarea datelor într-un mod elegant și declarativ, și pattern matching pentru expresivitate crescută. \parencite{charp}

De-a lungul timpului, C\# a evoluat constant, adăugând funcționalități moderne care îl fac competitiv cu cele mai populare limbaje de programare. Ultimele versiuni aduc îmbunătățiri semnificative în ceea ce privește performanța, claritatea sintaxei și siguranța codului. \parencite{charp}

Este adesea folosit în mediul enterprise datorită integrării excelente cu serviciile Microsoft (Azure, SQL Server, Active Directory etc.) și a ecosistemului .NET, care oferă o platformă stabilă, multiplatformă și bine întreținută. \parencite{charp}

Un aspect esențial în dezvoltarea aplicațiilor C\# este organizarea codului, iar aici intervin conceptele de design patterns, care oferă soluții reutilizabile pentru probleme comune de arhitectură software. \parencite{charp}

Una dintre cele mai utilizate structuri este separarea aplicației în straturi distincte: \textit{Repository} și \textit{Service}. Acest model este specific abordărilor precum Clean Architecture și Domain-Driven Design. \parencite{designPatterns}

Repository-ul este responsabil pentru interacțiunea cu sursa de date, fie că este o bază de date relațională, un storage NoSQL sau un API extern. El oferă o interfață clară pentru operații precum adăugarea, citirea, actualizarea sau ștergerea entităților, abstractizând detaliile concrete de stocare. \parencite{designPatterns}

Service-ul vine deasupra repository-ului și conține logica de business a aplicației. Aici sunt implementate regulile și fluxurile aplicației, validările și transformările de date. Service-ul folosește repository-ul ca sursă de date, dar nu se ocupă direct cu modul în care acestea sunt stocate sau obținute. \parencite{designPatterns}

\section{.NET}

.NET este o platformă de dezvoltare creată de Microsoft, concepută pentru a permite dezvoltatorilor să construiască aplicații moderne, sigure și performante pentru o varietate largă de dispozitive și sisteme. A fost lansată oficial în anul 2002 sub numele de .NET Framework, fiind destinată exclusiv sistemelor Windows. Scopul principal era de a oferi un cadru unitar, stabil și coerent în care dezvoltatorii să poată crea aplicații desktop și web într-un mod eficient, folosind limbaje precum C\# și VB.NET. \parencite{dotnet}

Pe măsură ce tehnologia a evoluat, Microsoft a început procesul de modernizare a platformei, orientându-se către dezvoltarea multiplatformă și open-source. Astfel, în 2016 a fost lansat .NET Core, o versiune rescrisă și modulară a platformei, capabilă să ruleze pe Windows, Linux și macOS. Aceasta a oferit o arhitectură mai flexibilă și un sistem de distribuție mai eficient, prin intermediul pachetelor NuGet. \parencite{dotnet}

Începând cu .NET 5, lansat în 2020, Microsoft a unificat direcțiile .NET Framework și .NET Core într-o singură platformă, denumită simplu .NET. Această versiune unificată continuă să evolueze, cu îmbunătățiri constante aduse în performanță, suportul pentru limbaje, instrumente de dezvoltare și capabilități cloud. Versiunile recente, cum ar fi .NET 6, .NET 7 și .NET 8, sunt toate versiuni LTS (Long Term Support) sau standard, folosite pe scară largă în industrie. \parencite{dotnet}

.NET se bazează pe concepte precum CLR (Common Language Runtime), care este motorul de execuție ce gestionează rularea aplicațiilor, și BCL (Base Class Library), o colecție vastă de librării care oferă funcționalități esențiale: lucrul cu fișiere, colecții, rețea, criptare, baze de date și multe altele. De asemenea, platforma oferă suport excelent pentru programarea asincronă, gestionarea memoriei automate prin Garbage Collection, precum și o integrare nativă cu servicii și tehnologii moderne precum Azure, Docker și Kubernetes. \parencite{dotnet}

Prin intermediul .NET, dezvoltatorii pot crea aplicații web (cu ASP.NET), desktop (cu WPF și WinForms), mobile (cu .NET MAUI și Xamarin), servicii API, aplicații cloud, jocuri (prin Unity) și chiar soluții de inteligență artificială sau machine learning. \parencite{dotnet}

\section{Azure Function}
Azure Functions este o platformă serverless dezvoltată de Microsoft care permite rularea codului în cloud fără a fi nevoie să gestionezi infrastructura de servere. Acest model facilitează dezvoltarea rapidă a aplicațiilor și serviciilor scalabile, concentrându-te doar pe logică, nu pe administrarea resurselor. \parencite{azureFunctions}
\\Principalele caracteristici ale Azure Functions sunt:
\begin{itemize}
    \item Execuție event-driven: Funcțiile sunt declanșate automat de evenimente, cum ar fi modificări în baza de date, mesaje din cozi, cereri HTTP, cronometre sau alte surse.
    \item Scalabilitate automată: Azure gestionează în mod automat scalarea funcțiilor în funcție de cerere, asigurând performanță optimă indiferent de volumul de trafic.
    \item Model de plată pay-as-you-go: Se plătește doar pentru timpul efectiv în care codul rulează, fără costuri fixe legate de infrastructură.
    \item Suport pentru mai multe limbaje: C\#, JavaScript, Python, Java, PowerShell și altele pot fi folosite pentru a scrie funcțiile.
\end{itemize}\parencite{azureFunctions}

\subsection{Azure Durable Functions}
Azure Durable Functions reprezintă o extensie a platformei Azure Functions, dezvoltată de Microsoft pentru a facilita crearea de fluxuri de lucru pe termen lung și orchestrarea funcțiilor serverless într-un mod eficient și scalabil. Lansată oficial în 2017, această extensie adaugă capabilități suplimentare funcțiilor serverless tradiționale, permițând dezvoltatorilor să gestioneze procese complexe care implică mai multe funcții interdependente și executate în timp. \parencite{azureDurableFunctions}

Azure Functions, lansat cu un an mai devreme (2016), face parte din suita de servicii serverless computing din Azure și permite rularea de cod fără a fi necesară administrarea explicită a infrastructurii. Totuși, aceste funcții standard au fost concepute pentru execuții rapide, de scurtă durată, ceea ce le făcea mai puțin potrivite pentru procesele care necesită menținerea stării și coordonarea pe termen lung. \parencite{azureDurableFunctions}
\\Pentru a răspunde acestor nevoi, Durable Functions oferă următoarele capabilități:
\begin{itemize}
    \item Orchestrarea funcțiilor de scurtă durată, într-un mod automatizat și declarativ;
    \item Gestionarea stării între apelurile funcțiilor, pe parcursul execuției unui proces complex;
    \item Retry automat și suport pentru scenarii de compensare în caz de eșec.
\end{itemize}\parencite{azureDurableFunctions}

Durable Functions este construit pe baza Durable Task Framework, permițând dezvoltatorilor să scrie cod orchestrat într-un stil secvențial, dar care este transformat automat în execuție asincronă și distribuită, cu păstrarea stării între pași. \parencite{azureDurableFunctions}
\\Scenarii comune de utilizare:
\begin{itemize}
    \item Orchestrarea fluxurilor de lucru
          O funcție orchestrator coordonează apelurile către alte funcții, în funcție de anumite condiții sau răspunsuri. De exemplu, poate apela o funcție care extrage date de la un API, apoi, în funcție de rezultat, lansează alte funcții în lanț.
    \item 	Function Chaining (lanțuri de funcții)
          Mai multe funcții sunt apelate secvențial, iar rezultatul fiecărei funcții este transmis mai departe către următoarea funcție din lanț.
    \item 	Fan-out/Fan-in
          O funcție poate declanșa mai multe funcții în paralel (fan-out), după care agregă rezultatele într-un punct comun (fan-in). Acest model este ideal pentru procesarea paralelă a unor seturi mari de date.
    \item 	Gestionarea proceselor de lungă durată
          Durable Functions permite execuția de fluxuri care pot dura ore, zile sau chiar luni, menținând starea între pași. Este util în scenarii precum aprobări, procese de onboarding, migrare de date sau procese distribuite în timp.
    \item 	Compensarea acțiunilor (Saga Pattern)
          În fluxuri unde mai multe acțiuni trebuie executate într-o ordine strictă, Durable Functions permite implementarea de logici de rollback sau compensare în caz de eșec, asigurând consistența procesului.
    \item 	Funcții temporizate (Timer Functions)
          Orchestratorii pot programa funcții să ruleze după o întârziere sau la anumite intervale. Acestea sunt utile în procese automate, monitorizări periodice sau trimiterea de notificări programate.
\end{itemize}\parencite{azureDurableFunctions}

Lucrarea folosește Azure Functions, în special Azure Durable Functions pentru a face posibilă procesarea și menținerea legăturii între 2 utilizatori pe parcursul cursei, indiferent de circumstațele actuale (utilizatorul face Refresh sau aplicația își ia Restart).

\section{Azure Table Storage}
Azure Table Storage este un serviciu creat de Microsoft ca parte a suitei Azure Storage, gândit pentru a oferi o soluție simplă, scalabilă și extrem de rapidă pentru stocarea datelor structurate în format NoSQL. Table Storage nu impune o schemă fixă, ceea ce înseamnă că fiecare entitate dintr-o tabelă poate avea un set diferit de proprietăți. Acest lucru îl face ideal pentru scenarii în care datele sunt variabile sau când flexibilitatea contează mai mult decât relațiile complexe dintre entități. \parencite{azureStorage}

A fost introdus în jurul anului 2010, într-o perioadă în care Microsoft își contura strategia cloud și începea să ofere servicii care să concureze cu AWS. A fost conceput pentru a servi aplicații la scară mare, care necesită o cantitate mare de scrieri și citiri, dar fără nevoia de tranzacții complexe sau relații între tabele. Scenariile clasice includ loguri, telemetrie, date despre utilizatori, mesaje sau orice tip de evenimente care trebuie stocate rapid și recuperate eficient. \parencite{azureStorage}

Table Storage folosește o structură bazată pe tabele care conțin entități, fiecare identificată printr-o combinație unică de PartitionKey și RowKey. Această combinație asigură performanță crescută la căutare, mai ales atunci când datele sunt distribuite corect în funcție de PartitionKey. Fiind un serviciu NoSQL, nu există conceptul de join-uri sau constrângeri între entități, dar în schimb se câștigă foarte mult la capitolul viteză și scalabilitate. \parencite{azureStorage}

În timp, Table Storage a devenit o alegere populară pentru aplicațiile moderne, în special cele distribuite, microservicii sau orice alt sistem care are nevoie de o bază de date simplă, ieftină și elastică. Este integrat profund în ecosistemul Azure, ceea ce înseamnă că poate fi combinat ușor cu alte servicii precum Azure Functions, Logic Apps sau Azure Event Grid, făcându-l extrem de versatil în arhitecturi cloud-native. \parencite{azureStorage}

Azure Storage reprezintă structura de stocare pentru lucrarea de față, datorită numărului scăzut de dependințe și tabele utilizate și vitezei de care dispune.

\subsection{Azurite}

Azurite este un emulator local creat de Microsoft pentru serviciile de stocare din Azure. A fost conceput ca un instrument de dezvoltare care le permite programatorilor să lucreze cu Azure Blob Storage, Queue Storage și Table Storage pe propriul calculator, fără a avea nevoie de o conexiune activă la platforma Azure. Practic, simulează comportamentul serviciilor reale, oferind o experiență de dezvoltare aproape identică cu mediul de producție. \parencite{azurite}

A apărut ca succesor al emulatorului mai vechi Microsoft Azure Storage Emulator, care era limitat la Windows. Azurite, în schimb, este cross-platform, fiind scris în Node.js și disponibil ca pachet NPM sau container Docker. Asta îl face ideal pentru proiecte care rulează pe macOS, Linux sau Windows, fie în linie de comandă, fie integrat în Visual Studio Code. \parencite{azurite}

Oferă un mediu de testare local rapid și fără costuri, în care dezvoltatorii pot simula citiri, scrieri, partajări și alerte, fără riscul de a consuma resurse reale în Azure sau de a introduce erori în datele live. E extrem de valoros în timpul dezvoltării și al testării automate, mai ales în scenarii unde aplicația interacționează frecvent cu Table Storage sau Blob-uri. \parencite{azurite}

Azurite ajută la accelerarea dezvoltării, reduce costurile și elimină nevoia de conexiune constantă la cloud, păstrând în același timp consistența cu ceea ce se întâmplă în Azure real. Din acest motiv este folosit și în proiectul curent.

\section{LINQ}
\section{HyperText Markup Language (HTML)}
\section{Cascading Style SHeets (CSS)}
\section{Bootstrap}
\section{Blazor}
Blazor este un framework open-source dezvoltat de Microsoft, lansat oficial în 2018, ca parte a ecosistemului .NET. Numele „Blazor” este format din cuvintele „Browser” și „Razor”, evidențiind utilizarea motorului Razor pentru redarea componentelor web direct în browser. \parencite{blazor}

Scopul principal al Blazor este de a permite dezvoltatorilor .NET să creeze aplicații web interactive fără a apela la JavaScript, oferind o alternativă la framework-uri front-end precum React, Angular sau Vue.js. Utilizând limbajul C\# și întreg ecosistemul .NET, Blazor a devenit rapid o opțiune atractivă pentru dezvoltatorii familiarizați cu tehnologiile Microsoft, facilitând dezvoltarea de aplicații full-stack doar cu .NET. \parencite{blazor}
\\Blazor este disponibil în două variante principale:
\begin{itemize}
    \item Blazor Server (2019): Aplicația rulează pe server, iar interacțiunea cu utilizatorul este gestionată în timp real prin SignalR. Această variantă oferă performanțe ridicate și un consum redus de resurse pe client, dar necesită o conexiune constantă la server.
    \item Blazor WebAssembly (2020): Codul C\# este compilat în WebAssembly și rulează direct în browser, eliminând nevoia unei conexiuni continue la server. Aceasta permite dezvoltarea de aplicații web care pot funcționa și offline.
\end{itemize}\parencite{blazor}
\\Utilizări principale ale Blazor:
\begin{itemize}
    \item Aplicații web interactive (SPA - Single Page Applications): Blazor permite dezvoltarea de aplicații de tip SPA, în care navigarea și interacțiunile cu utilizatorul se realizează fără reîncărcarea completă a paginii, oferind o experiență fluidă și modernă.
    \item Aplicații WebAssembly: Cu Blazor WebAssembly, aplicațiile pot rula complet în browser, reducând latențele și oferind posibilitatea de a crea aplicații offline sau cu funcționare locală.
    \item Aplicații server-side: Blazor Server este ideal pentru aplicații care necesită control sporit asupra datelor și un răspuns în timp real. Prin SignalR, modificările din UI sunt reflectate instant, fără apeluri repetate la server.
    \item Aplicații enterprise: Datorită integrării excelente cu ecosistemul .NET, Blazor este preferat în mediul enterprise pentru reutilizarea codului existent, integrarea ușoară a logicii de business, autentificării, bazelor de date și serviciilor API.
\end{itemize}\parencite{blazor}

\section{MudBlazor}

\section{SignalR}
 {\hspace*{1cm}WebSockets este un protocol de comunicație care permite o conexiune bidirecțională, persistentă și full-duplex între un client (de exemplu, un browser web) și un server, facilitând transmiterea rapidă și continuă a datelor în timp real fără a reîncarca pagina.}\parencite{signalR}\\
{\hspace*{1cm} SignalR este o bibliotecă dezvoltată de Microsoft care facilitează comunicarea în timp real între aplicațiile web, mobile sau desktop și servere. Lansată în 2011 și integrată ulterior în ecosistemul ASP.NET Core, SignalR permite actualizări și notificări instantanee fără a fi nevoie să reîncarci paginile web.}\parencite{signalR}\\
{Înainte de SignalR, comunicarea bidirecțională în timp real era dificilă și adesea implementată prin tehnici ineficiente precum polling sau long-polling, care consumau multe resurse. SignalR a simplificat acest proces prin integrarea automată a protocolului WebSockets, care oferă o conexiune persistentă și eficientă între client și server.}\parencite{signalR}\\
{Cu apariția ASP.NET Core, SignalR a fost reproiectat pentru a fi mai performant și scalabil, suportând diverse metode de transport și oferind o experiență optimă indiferent de mediu.}\parencite{signalR}\\
{SignalR folosește automat cel mai potrivit mecanism de comunicare, în funcție de capabilitățile clientului și ale serverului:}
\begin{itemize}
    \item WebSockets: Protocolul principal, oferind o conexiune rapidă și bidirecțională.
    \item Server-Sent Events (SSE): Permite serverului să trimită actualizări către client printr-o conexiune HTTP deschisă.
    \item Long Polling: Metoda de rezervă, în care clientul face cereri repetate pentru a verifica noutățile atunci când celelalte opțiuni nu sunt disponibile.
\end{itemize}\parencite{signalR}
{SignalR este ideal pentru aplicații care necesită actualizări în timp real, cum ar fi chat-uri, notificări, dashboard-uri live sau colaborare online.}
\subsection{SignalR Server Emulator}
\section{Refit API}
\section{Leaflet și Routing Machine}
\section{Stripe}
 {\hspace*{1cm} Stripe este o platformă globală de procesare a plăților, fondată în 2010 de frații Patrick și
  John Collison. Aceasta a fost creată pentru a permite afacerilor și dezvoltatorilor să accepte plăți
  online în mod simplu și sigur. Stripe a devenit rapid unul dintre cei mai populari furnizori de
  soluții de plăți digitale, datorită ușurinței de integrare, suportului pentru diverse metode de plată
  și disponibilității în multiple țări.}\parencite{stripe}\\
{\hspace*{1cm} Stripe a fost fondată într-o perioadă în care comerțul online era în creștere, dar soluțiile de
plată disponibile erau adesea greoaie sau complexe de implementat. Frații Collison au observat o
oportunitate de a crea o platformă care să facă procesul de integrare a plăților în site-uri web și
aplicații mult mai simplu pentru dezvoltatori. Stripe a fost lansată oficial în 2011, cu misiunea de
a moderniza plățile online și de a oferi soluții intuitive pentru afaceri de toate dimensiunile.}\parencite{stripe}