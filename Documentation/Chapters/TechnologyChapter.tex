\chapter{TEHNOLOGII ȘI FRAMEWORK-URI FOLOSITE}
\section{C\#}

\section{.NET}
\section{Azure Function}
Azure Functions este o platformă serverless dezvoltată de Microsoft care permite rularea codului în cloud fără a fi nevoie să gestionezi infrastructura de servere. Acest model facilitează dezvoltarea rapidă a aplicațiilor și serviciilor scalabile, concentrându-te doar pe logică, nu pe administrarea resurselor. \parencite{azureFunctions}
\\Principalele caracteristici ale Azure Functions sunt:
\begin{itemize}
    \item Execuție event-driven: Funcțiile sunt declanșate automat de evenimente, cum ar fi modificări în baza de date, mesaje din cozi, cereri HTTP, cronometre sau alte surse.
    \item Scalabilitate automată: Azure gestionează în mod automat scalarea funcțiilor în funcție de cerere, asigurând performanță optimă indiferent de volumul de trafic.
    \item Model de plată pay-as-you-go: Se plătește doar pentru timpul efectiv în care codul rulează, fără costuri fixe legate de infrastructură.
    \item Suport pentru mai multe limbaje: C\#, JavaScript, Python, Java, PowerShell și altele pot fi folosite pentru a scrie funcțiile.
\end{itemize}\parencite{azureFunctions}

\subsection{Azure Durable Functions}
Azure Durable Functions reprezintă o extensie a platformei Azure Functions, dezvoltată de Microsoft pentru a facilita crearea de fluxuri de lucru pe termen lung și orchestrarea funcțiilor serverless într-un mod eficient și scalabil. Lansată oficial în 2017, această extensie adaugă capabilități suplimentare funcțiilor serverless tradiționale, permițând dezvoltatorilor să gestioneze procese complexe care implică mai multe funcții interdependente și executate în timp. \parencite{azureDurableFunctions}

Azure Functions, lansat cu un an mai devreme (2016), face parte din suita de servicii serverless computing din Azure și permite rularea de cod fără a fi necesară administrarea explicită a infrastructurii. Totuși, aceste funcții standard au fost concepute pentru execuții rapide, de scurtă durată, ceea ce le făcea mai puțin potrivite pentru procesele care necesită menținerea stării și coordonarea pe termen lung. \parencite{azureDurableFunctions}
\\Pentru a răspunde acestor nevoi, Durable Functions oferă următoarele capabilități:
\begin{itemize}
    \item Orchestrarea funcțiilor de scurtă durată, într-un mod automatizat și declarativ;
    \item Gestionarea stării între apelurile funcțiilor, pe parcursul execuției unui proces complex;
    \item Retry automat și suport pentru scenarii de compensare în caz de eșec.
\end{itemize}\parencite{azureDurableFunctions}

Durable Functions este construit pe baza Durable Task Framework, permițând dezvoltatorilor să scrie cod orchestrat într-un stil secvențial, dar care este transformat automat în execuție asincronă și distribuită, cu păstrarea stării între pași. \parencite{azureDurableFunctions}
\\Scenarii comune de utilizare:
\begin{itemize}
    \item Orchestrarea fluxurilor de lucru
          O funcție orchestrator coordonează apelurile către alte funcții, în funcție de anumite condiții sau răspunsuri. De exemplu, poate apela o funcție care extrage date de la un API, apoi, în funcție de rezultat, lansează alte funcții în lanț.
    \item 	Function Chaining (lanțuri de funcții)
          Mai multe funcții sunt apelate secvențial, iar rezultatul fiecărei funcții este transmis mai departe către următoarea funcție din lanț.
    \item 	Fan-out/Fan-in
          O funcție poate declanșa mai multe funcții în paralel (fan-out), după care agregă rezultatele într-un punct comun (fan-in). Acest model este ideal pentru procesarea paralelă a unor seturi mari de date.
    \item 	Gestionarea proceselor de lungă durată
          Durable Functions permite execuția de fluxuri care pot dura ore, zile sau chiar luni, menținând starea între pași. Este util în scenarii precum aprobări, procese de onboarding, migrare de date sau procese distribuite în timp.
    \item 	Compensarea acțiunilor (Saga Pattern)
          În fluxuri unde mai multe acțiuni trebuie executate într-o ordine strictă, Durable Functions permite implementarea de logici de rollback sau compensare în caz de eșec, asigurând consistența procesului.
    \item 	Funcții temporizate (Timer Functions)
          Orchestratorii pot programa funcții să ruleze după o întârziere sau la anumite intervale. Acestea sunt utile în procese automate, monitorizări periodice sau trimiterea de notificări programate.
\end{itemize}\parencite{azureDurableFunctions}

Lucrarea folosește Azure Functions, în special Azure Durable Functions pentru a face posibilă procesarea și menținerea legăturii între 2 utilizatori pe parcursul cursei, indiferent de circumstațele actuale (utilizatorul face Refresh sau aplicația își ia Restart).

\section{Azure Table Storage}
\subsection{Azurite}
\section{LINQ}
\section{HyperText Markup Language (HTML)}
\section{Cascading Style SHeets (CSS)}
\section{Bootstrap}
\section{Blazor}
Blazor este un framework open-source dezvoltat de Microsoft, lansat oficial în 2018, ca parte a ecosistemului .NET. Numele „Blazor” este format din cuvintele „Browser” și „Razor”, evidențiind utilizarea motorului Razor pentru redarea componentelor web direct în browser. \parencite{blazor}

Scopul principal al Blazor este de a permite dezvoltatorilor .NET să creeze aplicații web interactive fără a apela la JavaScript, oferind o alternativă la framework-uri front-end precum React, Angular sau Vue.js. Utilizând limbajul C\# și întreg ecosistemul .NET, Blazor a devenit rapid o opțiune atractivă pentru dezvoltatorii familiarizați cu tehnologiile Microsoft, facilitând dezvoltarea de aplicații full-stack doar cu .NET. \parencite{blazor}
\\Blazor este disponibil în două variante principale:
\begin{itemize}
    \item Blazor Server (2019): Aplicația rulează pe server, iar interacțiunea cu utilizatorul este gestionată în timp real prin SignalR. Această variantă oferă performanțe ridicate și un consum redus de resurse pe client, dar necesită o conexiune constantă la server.
    \item Blazor WebAssembly (2020): Codul C\# este compilat în WebAssembly și rulează direct în browser, eliminând nevoia unei conexiuni continue la server. Aceasta permite dezvoltarea de aplicații web care pot funcționa și offline.
\end{itemize}\parencite{blazor}
\\Utilizări principale ale Blazor:
\begin{itemize}
    \item Aplicații web interactive (SPA - Single Page Applications): Blazor permite dezvoltarea de aplicații de tip SPA, în care navigarea și interacțiunile cu utilizatorul se realizează fără reîncărcarea completă a paginii, oferind o experiență fluidă și modernă.
    \item Aplicații WebAssembly: Cu Blazor WebAssembly, aplicațiile pot rula complet în browser, reducând latențele și oferind posibilitatea de a crea aplicații offline sau cu funcționare locală.
    \item Aplicații server-side: Blazor Server este ideal pentru aplicații care necesită control sporit asupra datelor și un răspuns în timp real. Prin SignalR, modificările din UI sunt reflectate instant, fără apeluri repetate la server.
    \item Aplicații enterprise: Datorită integrării excelente cu ecosistemul .NET, Blazor este preferat în mediul enterprise pentru reutilizarea codului existent, integrarea ușoară a logicii de business, autentificării, bazelor de date și serviciilor API.
\end{itemize}\parencite{blazor}

\section{MudBlazor}

\section{SignalR}
 {\hspace*{1cm}WebSockets este un protocol de comunicație care permite o conexiune bidirecțională, persistentă și full-duplex între un client (de exemplu, un browser web) și un server, facilitând transmiterea rapidă și continuă a datelor în timp real fără a reîncarca pagina.}\parencite{signalR}\\
{\hspace*{1cm} SignalR este o bibliotecă dezvoltată de Microsoft care facilitează comunicarea în timp real între aplicațiile web, mobile sau desktop și servere. Lansată în 2011 și integrată ulterior în ecosistemul ASP.NET Core, SignalR permite actualizări și notificări instantanee fără a fi nevoie să reîncarci paginile web.}\parencite{signalR}\\
{Înainte de SignalR, comunicarea bidirecțională în timp real era dificilă și adesea implementată prin tehnici ineficiente precum polling sau long-polling, care consumau multe resurse. SignalR a simplificat acest proces prin integrarea automată a protocolului WebSockets, care oferă o conexiune persistentă și eficientă între client și server.}\parencite{signalR}\\
{Cu apariția ASP.NET Core, SignalR a fost reproiectat pentru a fi mai performant și scalabil, suportând diverse metode de transport și oferind o experiență optimă indiferent de mediu.}\parencite{signalR}\\
{SignalR folosește automat cel mai potrivit mecanism de comunicare, în funcție de capabilitățile clientului și ale serverului:}
\begin{itemize}
    \item WebSockets: Protocolul principal, oferind o conexiune rapidă și bidirecțională.
    \item Server-Sent Events (SSE): Permite serverului să trimită actualizări către client printr-o conexiune HTTP deschisă.
    \item Long Polling: Metoda de rezervă, în care clientul face cereri repetate pentru a verifica noutățile atunci când celelalte opțiuni nu sunt disponibile.
\end{itemize}\parencite{signalR}
{SignalR este ideal pentru aplicații care necesită actualizări în timp real, cum ar fi chat-uri, notificări, dashboard-uri live sau colaborare online.}
\subsection{SignalR Server Emulator}
\section{Refit API}
\section{Leaflet și Routing Machine}
\section{Stripe}
 {\hspace*{1cm} Stripe este o platformă globală de procesare a plăților, fondată în 2010 de frații Patrick și
  John Collison. Aceasta a fost creată pentru a permite afacerilor și dezvoltatorilor să accepte plăți
  online în mod simplu și sigur. Stripe a devenit rapid unul dintre cei mai populari furnizori de
  soluții de plăți digitale, datorită ușurinței de integrare, suportului pentru diverse metode de plată
  și disponibilității în multiple țări.}\parencite{stripe}\\
{\hspace*{1cm} Stripe a fost fondată într-o perioadă în care comerțul online era în creștere, dar soluțiile de
plată disponibile erau adesea greoaie sau complexe de implementat. Frații Collison au observat o
oportunitate de a crea o platformă care să facă procesul de integrare a plăților în site-uri web și
aplicații mult mai simplu pentru dezvoltatori. Stripe a fost lansată oficial în 2011, cu misiunea de
a moderniza plățile online și de a oferi soluții intuitive pentru afaceri de toate dimensiunile.}\parencite{stripe}