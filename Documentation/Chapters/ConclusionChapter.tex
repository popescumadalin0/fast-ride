\chapter{CONCLUZII}
Lucrarea a avut ca obiectiv dezvoltarea unei aplicații de tip ride-sharing, FastRide, care gestionează într-un mod eficient relația dintre clienți și șoferi. Aplicația pune accent pe simplitate, stabilitate și integrare între mai multe tehnologii actuale, reușind să acopere un flux funcțional complet: de la autentificarea utilizatorului până la vizualizarea unei curse finalizate.

Pe parcursul realizării proiectului, s-a urmărit nu doar implementarea funcționalităților de bază, ci și înțelegerea a modului în care diferite componente software pot comunica între ele. Utilizarea Blazor WebAssembly a oferit o experiență fluidă în browser, în timp ce Azure Durable Functions a permis gestionarea logicii din spatele aplicației într-un mod scalabil și bine structurat. Componentele precum SignalR, Leaflet, Stripe și integrarea cu contul Google au completat la integritatea aplicației, transformând-o într-un produs și serviciu pregătit pentru utilizarea în producție.

Un alt obiectiv atins a fost acela de a dezvolta o aplicație ușor de extins. Arhitectura aleasă permite adăugarea de noi funcționalități, precum potențiale îmbunătățiri în zona de comunicare între utilizatori, istoric detaliat al curselor sau notificări avansate.

Pe lângă dezvoltarea tehnică, acest proiect a contribuit semnificativ la consolidarea cunoștințelor privind dezvoltarea aplicațiilor web, interacțiunea cu serviciile cloud, organizarea codului și gestionarea datelor. A reprezentat o ocazie bună de a lucra cu tehnologii moderne într-un scenariu real, apropiat de cerințele industriei.

În concluzie, Fast Ride nu este doar o lucrare academică, ci și un exemplu de proiect funcțional, care poate servi ca punct de plecare pentru dezvoltări viitoare și care reflectă atât efortul tehnic depus, cât și dorința de a construi aplicații utile, stabile și ușor de folosit.