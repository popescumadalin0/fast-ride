\chapter{INTRODUCERE}
\section{Scopul principal}
Fast Ride este o aplicație de ride-sharing, concepută pentru a facilita legătura directă între clienți și șoferi, într-un mod rapid și eficient.

Scopul principal al aplicației Fast Ride este de a oferi un mediu online prin care utilizatorii (clienți) pot cere curse în timp real, iar șoferii disponibili le pot accepta într-un mod simplu și intuitiv. Aplicația urmărește să automatizeze complet procesul de conectare dintre cererea și oferta de transport urban, eliminând nevoia de apeluri telefonice.

\section{Motivație}
Tema aleasă pentru această lucrare pornește de la nevoia pentru accesul rapid și eficient la servicii de transport. Odată cu dezvoltarea orașelor și creșterea traficului, s-a observat o tendință accentuată spre utilizarea aplicațiilor de tip ride-sharing, care oferă o alternativă flexibilă la transportul clasic.

Aplicația Fast Ride propune o soluție simplă și eficientă pentru gestionarea curselor între clienți și șoferi. Ideea de a dezvolta această aplicație a venit din dorința de a construi un proiect complet, în care să se regăsească elemente din tot ce înseamnă dezvoltare software modernă: o interfață prietenoasă, comunicare în timp real, stocare în cloud și integrare cu servicii externe.

Tehnologiile alese: Blazor WebAssembly pentru partea de frontend, Azure Durable Functions pentru backend și SignalR pentru actualizări live, au permis realizarea unei aplicații distribuite, capabile să răspundă în timp real cerințelor utilizatorilor. De asemenea, integrarea cu Stripe pentru validarea cardurilor și autentificarea prin cont Google au contribuit la crearea unei experiențe cât mai fluide și sigure.

Prin aceasta lucrare mi-am propus dezvoltarea unei aplicații funcționale, care poate fi ușor extinsă și adaptată în viitor, dar și învățarea unor tehnologii moderne folosite în proiecte reale.